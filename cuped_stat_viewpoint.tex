\documentclass[10pt, a4paper]{article}

\usepackage{etex} % расширение классического tex
% в частности позволяет подгружать гораздо больше пакетов, чем мы и займёмся далее

%\usepackage{mathtext} % русские буквы в формулах? (и без неё работает)
% Например, $x_{\text{один}}$


\usepackage{verbatim} % для многострочных комментариев
\usepackage{makeidx} % для создания предметных указателей



\usepackage{setspace}
\usepackage{amsmath, amsfonts, amssymb, amsthm}
\usepackage{mathrsfs} % sudo yum install texlive-rsfs
\usepackage{dsfont} % sudo yum install texlive-doublestroke
\usepackage{array, multicol, multirow, bigstrut} % sudo yum install texlive-multirow
\usepackage{indentfirst} % установка отступа в первом абзаце главы

\usepackage{physics} % много вкусного


\usepackage{bm}
\usepackage{bbm} % шрифт с двойными буквами
%\usepackage[perpage]{footmisc}

\usepackage{dcolumn} % центрирование по разделителю для apsrtable

% создание гиперссылок в pdf
\usepackage[unicode, colorlinks=true, urlcolor=blue, hyperindex, breaklinks]{hyperref}


\usepackage{microtype} % свешиваем пунктуацию
% теперь знаки пунктуации могут вылезать за правую границу текста, при этом текст выглядит ровнее


\usepackage{textcomp}  % Чтобы в формулах можно было русские буквы писать через \text{}

% размер листа бумаги
%\usepackage[paperwidth=145mm,paperheight=215mm,
%height=182mm,width=113mm,top=20mm,includefoot]%{geometry}
\usepackage[paper=a4paper, top=15mm, bottom=13.5mm, left=16.5mm, right=13.5mm, includefoot]{geometry}

\usepackage{xcolor}

% \usepackage[pdftex]{graphicx} % для вставки графики, убрано, т.к. knitr похоже сам добавляет

\usepackage{float, longtable}
\usepackage{soulutf8}

\usepackage{enumitem} % дополнительные плюшки для списков
%  например \begin{enumerate}[resume] позволяет продолжить нумерацию в новом списке

\usepackage{mathtools}
\usepackage{cancel, xspace} % sudo yum install texlive-cancel

% \usepackage{minted} % display program code with syntax highlighting
% требует установки pygments и python

\usepackage{numprint} % sudo yum install texlive-numprint
\npthousandsep{,}\npthousandthpartsep{}\npdecimalsign{.}


\usepackage{subfigure} % для создания нескольких рисунков внутри одного

\usepackage{tikz, pgfplots} % язык для рисования графики из latex'a
\pgfplotsset{compat=1.16}
\usetikzlibrary{trees} % tikz-прибамбас для рисовки деревьев
\usepackage{tikz-qtree} % альтернативный tikz-прибамбас для рисовки деревьев
\usetikzlibrary{arrows} % tikz-прибамбас для рисовки стрелочек подлиннее

\usepackage{todonotes} % для вставки в документ заметок о том, что осталось сделать
% \todo{Здесь надо коэффициенты исправить}
% \missingfigure{Здесь будет Последний день Помпеи}
% \listoftodos --- печатает все поставленные \todo'шки



\usepackage{booktabs} %  красивые таблицы
% заповеди из докупентации:
% 1. Не используйте вертикальные линни
% 2. Не используйте двойные линии
% 3. Единицы измерения - в шапку таблицы
% 4. Не сокращайте .1 вместо 0.1
% 5. Повторяющееся значение повторяйте, а не говорите "то же"


\usepackage{fontspec} % что-то про шрифты?
\usepackage{polyglossia} % русификация xelatex
\usepackage{csquotes}

\setmainlanguage{english}
\setotherlanguages{english}

% download "Linux Libertine" fonts:
% http://www.linuxlibertine.org/index.php?id=91&L=1
\setmainfont{Linux Libertine O} % or Helvetica, Arial, Cambria
% why do we need \newfontfamily:
% http://tex.stackexchange.com/questions/91507/
\newfontfamily{\cyrillicfonttt}{Linux Libertine O}

\AddEnumerateCounter{\asbuk}{\russian@alph}{щ} % для списков с русскими буквами
\setlist[enumerate, 1]{label=\asbuk*),ref=\asbuk*}





% Операторы
\DeclareMathOperator*\plim{plim}
\DeclareMathOperator{\sgn}{sign}
\DeclareMathOperator{\sign}{sign}
\DeclareMathOperator*{\argmin}{arg\,min}
\DeclareMathOperator*{\argmax}{arg\,max}
\DeclareMathOperator*{\amn}{arg\,min}
\DeclareMathOperator*{\amx}{arg\,max}
\DeclareMathOperator{\cov}{Cov}
\DeclareMathOperator{\Var}{Var}
\DeclareMathOperator{\Cov}{Cov}
\DeclareMathOperator{\Corr}{Corr}
\DeclareMathOperator{\pCorr}{pCorr}
\DeclareMathOperator{\E}{\mathbb{E}}
\let\P\relax
\DeclareMathOperator{\P}{\mathbb{P}}
\renewcommand{\le}{\leqslant}
\renewcommand{\ge}{\geqslant}
\renewcommand{\leq}{\leqslant}
\renewcommand{\geq}{\geqslant}

% Распределения
\newcommand{\cN}{\mathcal{N}}
\newcommand{\cU}{\mathcal{U}}
\newcommand{\cBinom}{\mathcal{Binom}}
\newcommand{\cPois}{\mathcal{Pois}}
\newcommand{\cBeta}{\mathcal{Beta}}
\newcommand{\cGamma}{\mathcal{Gamma}}

% Множества
\def \R{\mathbb{R}}
\def \N{\mathbb{N}}
\def \Z{\mathbb{Z}}

% Другое
\newcommand{\dx}[1]{\,\mathrm{d}#1} % Для интеграла: маленький отступ и прямая d
\newcommand{\ind}[1]{\mathbbm{1}_{\{#1\}}} % Индикатор события
\newcommand{\iid}{\mathrel{\stackrel{\rm i.\,i.\,d.}\sim}}
\newcommand{\const}{\mathrm{const}}

\title{CUPED: statistician viewpoint}
\author{Boris Demeshev}

\begin{document}

\begin{abstract}
    I try to give honest statistical background to the CUPED method.
    This allows us to correctly include multiple predictors and 
    use heteroscedasticity robust standard errors.
\end{abstract}


\maketitle

\section{Déjà vu}

On the third page Deng writes ‘the linear model makes strong assumptions that
are usually not satisfied in practice, i.e., the conditional expectation of the outcome metric is linear in the treatment
assignment and covariates. In addition, it also requires all
residuals to have a common variance’.

As I am teaching statistics and econometrics I was eager to read further. 
But than I encounter $\theta = \Cov(Y, X) / \Var(X)$ in equation 4 which is a theoretical
counterpart of slope estimate in simple regression. 
And later t-test is applied to $\Delta_{cv}$ that is again equivalent to a second simple regression.
Regression is replaced by something similar to two regressions. Déjà vu.

So I decided to expose the CUPED method using old boring regression language. 
Let's see what will hapen!

\section{Old regression friend}

To simplify the use of regression language I will start with one dataset of $n$ observations
with three variables:

\begin{itemize}
    \item $w_i$ the indicator of treatment: $w_i = 1$ for the treated group and $w_i=0$ for 
    the untreated group.
    \item $x_i$ any covariate that is a-priori independent with treatment indicator $w_i$.
    \item $y_i$ the target variable that is probably dependent both with $w_i$ and 
    $x_i$. 
\end{itemize}

Using regression language CUPED is a two step procedure:

\textbf{Step 1a}. Estimate the following regression using OLS: 
\[
    \hat y_i = \hat \gamma_1 + \hat\gamma_2 w_i + \hat \theta x_i.
\]

\textbf{Step 1b}. Calculate the semiresidual $r_i = y_i - \hat \theta x_i$.

I call this $r_i$ 'semiresudual' as classic residual in econometrics is 
\[
\hat u_i = y_i - \hat y_i = y_i - (\hat\gamma_1 + \hat\gamma_2 w_i + \hat\theta x_i).
\]


\textbf{Step 2a}. Estimate the second regression using OLS:
\[
    \hat r_i = \hat \beta_1 + \hat \beta_2 w_i.
\]

\textbf{Step 2b}. Use classical standard errors to build confidence interval for $\beta_2$.


Why this two-step procedure is better than just plain old multivariate regression
\[
    \hat y_i = \hat \gamma_1 + \hat\gamma_2 w_i + \hat \theta x_i
\]
with confidence interval for $\gamma_2$ build with classic standard errors? 



Honestly speaking Deng is not very explicit which regression should be used in the first step.  
On the page three the theoretical unknown $\theta$ is used. 

So one may also consider a simplier alternative regression
\[
    \hat y_i = \hat \gamma_1 + \hat \theta x_i.
\]
I will discuss why I prefer the inclusion of $w_i$ as regressor in the first step.




\section{Comparison with multivariate regression}

Let's talk about numeric estimates withoud assumptions at all. 

[the proof of equality]






Let's start easy first. No heteroscedasticity and no interaction between treatment $w_i$ and covariate $x_i$.
Correctly specified linear model. 

Assume that the true model is 
\[
y_i = \gamma_1 + \gamma_2 w_i + \theta x_i + u_i.
\]

The observations are independent and identically distributed with finite forth moments. 
The error term $u_i$ satisfies $\E(u_i \mid X) = 0$, $\Var(u_i \mid X) = \sigma^2$.

[here goes the picture]


It is well known in econometrics that OLS estimator $\hat\gamma_2$ is unbiased and consistent in this case.
As $\hat\beta_2$ estimate from second step is exactly equal to $\hat\gamma_2$ the same result applies.

And what about standard errors?







\section{Toy problem to understand the difference}




\section{Heteroscedasticity case}


\section{Unanswered questions}




\end{document}